\documentclass{bmcart}

%%%%%%%%%%%%%%%%%%%%%%%%%%%%%%%%%%%%%%%%%%%%%%
%%                                          %%
%% CARGA DE PAQUETES DE LATEX               %%
%%                                          %%
%%%%%%%%%%%%%%%%%%%%%%%%%%%%%%%%%%%%%%%%%%%%%%

%%% Load packages
\usepackage{amsthm,amsmath}
\usepackage{graphicx}
%\RequirePackage[numbers]{natbib}
%\RequirePackage{hyperref}
\usepackage[utf8]{inputenc} %unicode support
%\usepackage[applemac]{inputenc} %applemac support if unicode package fails
%\usepackage[latin1]{inputenc} %UNIX support if unicode package fails


%%%%%%%%%%%%%%%%%%%%%%%%%%%%%%%%%%%%%%%%%%%%%%
%%                                          %%
%% COMIENZO DEL DOCUMENTO                   %%
%%                                          %%
%%%%%%%%%%%%%%%%%%%%%%%%%%%%%%%%%%%%%%%%%%%%%%

\begin{document}

	\begin{frontmatter}
	
		\begin{fmbox}
			\dochead{Research}
			
			%%%%%%%%%%%%%%%%%%%%%%%%%%%%%%%%%%%%%%%%%%%%%%
			%% INTRODUCIR TITULO PROYECTO               %%
			%%%%%%%%%%%%%%%%%%%%%%%%%%%%%%%%%%%%%%%%%%%%%%
			
			\title{A sample article title}
			
			%%%%%%%%%%%%%%%%%%%%%%%%%%%%%%%%%%%%%%%%%%%%%%
			%% AUTORES. METER UNA ENTRADA AUTHOR        %%
			%% POR PERSONA                              %%
			%%%%%%%%%%%%%%%%%%%%%%%%%%%%%%%%%%%%%%%%%%%%%%
			
			\author[
			  addressref={aff1},                   % ESTA LINEA SE COPIA IGUAL PARA CADA AUTOR
			  corref={aff1},                       % ESTA LINEA SOLO DEBE TENERLA EL COORDINADOR DEL GRUPO
			  email={jane.e.doe@cambridge.co.uk}   % VUESTRO CORREO ACTIVO
			]{\inits{J.E.}\fnm{Jane E.} \snm{Doe}} % inits: INICIALES DE AUTOR, fnm: NOMBRE DE AUTOR, snm: APELLIDOS DE AUTOR
			\author[
			  addressref={aff1},
			  email={john.RS.Smith@cambridge.co.uk}
			]{\inits{J.R.S.}\fnm{John R.S.} \snm{Smith}}
			
			%%%%%%%%%%%%%%%%%%%%%%%%%%%%%%%%%%%%%%%%%%%%%%
			%% AFILIACION. NO TOCAR                     %%
			%%%%%%%%%%%%%%%%%%%%%%%%%%%%%%%%%%%%%%%%%%%%%%
			
			\address[id=aff1]{%                           % unique id
			  \orgdiv{ETSI Informática},             % department, if any
			  \orgname{Universidad de Málaga},          % university, etc
			  \city{Málaga},                              % city
			  \cny{España}                                    % country
			}
		
		\end{fmbox}% comment this for two column layout
		
		\begin{abstractbox}
		
			\begin{abstract} % abstract
			
			%%%%%%%%%%%%%%%%%%%%%%%%%%%%%%%%%%%%%%%%%%%%%%%
			%% RESUMEN BREVE DE NO MAS DE 100 PALABRAS   %%
			%%%%%%%%%%%%%%%%%%%%%%%%%%%%%%%%%%%%%%%%%%%%%%%	
			
			\end{abstract}
			
			%%%%%%%%%%%%%%%%%%%%%%%%%%%%%%%%%%%%%%%%%%%%%%
			%% PALABRAS CLAVE DEL PROYECTO              %%
			%%%%%%%%%%%%%%%%%%%%%%%%%%%%%%%%%%%%%%%%%%%%%%
			
			\begin{keyword}
			\kwd{sample}
			\kwd{article}
			\kwd{author}
			\end{keyword}
		
		
		\end{abstractbox}
	
	\end{frontmatter}
	
	%%%%%%%%%%%%%%%%%%%%%%%%%%%%%%%%%%%%%%%%%%%%%%%%%%%%%%%%%%%%%%%%%%%%%%%%%%%%%%%%%%%%%%%%
	%% EJEMPLO DE LATEX %%                                                                %%
	%% BORRAR ANTES DE ENTREGAR!!!!!!!!!!!!!!!!!!!!!!!!!!!!!!!!!!!!!!!!!!!!!              %%
	%%%%%%%%%%%%%%%%%%%%%%%%%%%%%%%%%%%%%%%%%%%%%%%%%%%%%%%%%%%%%%%%%%%%%%%%%%%%%%%%%%%%%%%%
	
	\section*{Content}
		Text and results for this section, as per the individual journal's instructions for authors. Here, we reference the figure \ref{fig:cost_genome} and figure \ref{fig:cost_megabase} but also the table \ref{tab:ejemplo}.
	
	\section*{Section title}
		Text for this section\ldots

		In this section we examine the growth rate of the mean of $Z_0$, $Z_1$ and $Z_2$. In
		addition, we examine a common modeling assumption and note the
		importance of considering the tails of the extinction time $T_x$ in
		studies of escape dynamics.
		We will first consider the expected resistant population at $vT_x$ for
		some $v>0$, (and temporarily assume $\alpha=0$)
		%
		\[
		E \bigl[Z_1(vT_x) \bigr]=
		\int_0^{v\wedge
			1}Z_0(uT_x)
		\exp (\lambda_1)\,du .
		\]
		%
		If we assume that sensitive cells follow a deterministic decay
		$Z_0(t)=xe^{\lambda_0 t}$ and approximate their extinction time as
		$T_x\approx-\frac{1}{\lambda_0}\log x$, then we can heuristically
		estimate the expected value as
		%
		\begin{equation}\label{eqexpmuts}
			\begin{aligned}[b]
				&      E\bigl[Z_1(vT_x)\bigr]\\
				&\quad      = \frac{\mu}{r}\log x
				\int_0^{v\wedge1}x^{1-u}x^{({\lambda_1}/{r})(v-u)}\,du .
			\end{aligned}
		\end{equation}
		%
		%%%%%%%%%%%%%%%%%%%%%%%%%%%%%%%%%%%%%%%%%%%%%%%%%%%%%%%%%%%%%%%%%%%%%%
		%% USAR \cite{...} PARA INCLUIR REFERENCIAS BIBLIOGRAFICAS          %%
		%%  \cite{koon}  Para una sola                                      %%
		%%  \cite{oreg,khar,zvai,xjon,schn,pond} Para una lista             %%
		%%%%%%%%%%%%%%%%%%%%%%%%%%%%%%%%%%%%%%%%%%%%%%%%%%%%%%%%%%%%%%%%%%%%%%
		Thus we observe that this expected value is finite for all $v>0$ (also see \cite{koon,xjon,marg,schn,koha,issnic}).
		
		
		%%%%%%%%%%%%%%%%%%%%%%%%%%%%%%%%%%%%%%%%%%%%%%%%%%%%%%%%%%%%%%%%%%%%%%%%%%%%%%%%%%%%%%%%%%%
		%% FIGURAS                                                                               %%
		%% includegraphics: inserta la imagen                                                    %%
		%% caption: descripcion de la figura                                                     %%
		%% label: etiqueta para hacer referencia a la figura en el texto con la instrucción ref  %%
		%%%%%%%%%%%%%%%%%%%%%%%%%%%%%%%%%%%%%%%%%%%%%%%%%%%%%%%%%%%%%%%%%%%%%%%%%%%%%%%%%%%%%%%%%%%	
		
		\begin{figure}[h!]
			\includegraphics[width=0.9\textwidth]{figures/Sequencing_Cost_per_Genome_May2020.jpg}
			\caption{Sample figure title}
			\label{fig:cost_genome}
		\end{figure}
		
		\begin{figure}[h!]
			\includegraphics[width=0.9\textwidth]{figures/Sequencing_Cost_per_Megabase_May2020.jpg}
			\caption{Sample figure title}
			\label{fig:cost_megabase}
		\end{figure}
		
		%%%%%%%%%%%%%%%%%%%%%%%%%%%%%%%%%%%%%%%%%%%%%%%%%%%%%%%%%%%%%%%%%%%%%%%%%%%%%%%%%%%%%%%%%%
		%% TABLAS                                                                               %%
		%% caption: Descripción tabla                                                           %%
		%% \begin{tabular}{letras}: Indica numero de columnas.                                  %%
		%%    Una letra por columna, la letra indica la alineación de la columna:               %%
		%%    c center, l left, r right                                                         %%
		%% hline: Representa una linea como entre filas                                         %%
		%% \\: fin de fila                                                                      %%
		%% &: delimitador de celda                                                              %%
		%% label: etiqueta para hacer referencia a la tabla en el texto con la instrucción ref  %%
		%%%%%%%%%%%%%%%%%%%%%%%%%%%%%%%%%%%%%%%%%%%%%%%%%%%%%%%%%%%%%%%%%%%%%%%%%%%%%%%%%%%%%%%%%%
		
		\begin{table}[h!]
			\caption{Sample table title. This is where the description of the table should go}
			\begin{tabular}{cccc}
				\hline
				& B1  &B2   & B3\\ 
				\hline
				A1 & 0.1 & 0.2 & 0.3\\
				A2 & ... & ..  & .\\
				A3 & ..  & .   & .\\ 
				\hline
				\label{tab:ejemplo}
			\end{tabular}
		\end{table}
				
		\subsection*{Sub-heading for section}
			Text for this sub-heading\ldots
	
			\subsubsection*{Sub-sub heading for section}
				Text for this sub-sub-heading\ldots
					
					\paragraph*{Sub-sub-sub heading for section}
						Text for this sub-sub-sub-heading\ldots
	
	%%%%%%%%%%%%%%%%%%%%%%%%%%%%%%%%%
	%% FIN DE EJEMPLO !!!!!!!!!!!! %%
	%%%%%%%%%%%%%%%%%%%%%%%%%%%%%%%%%
	
	%%%%%%%%%%%%%%%%%%%%%%%%%%%%%%%%%
	%% COMIENZO DEL DOCUMENTO REAL %%
	%%%%%%%%%%%%%%%%%%%%%%%%%%%%%%%%%
	
	
\section{Introducción}

Las enfermedades neurodegenerativas constituyen uno de los mayores desafíos biomédicos contemporáneos debido a su alta prevalencia y a la ausencia de tratamientos curativos eficaces \cite{Kelser2024} . Aunque sus manifestaciones clínicas son diversas, comparten un rasgo común: la degeneración neurológica progresiva, entendida como un deterioro gradual e irreversible de las funciones neuronales que afecta al movimiento, la cognición y la función sensorial \cite{Gao2008} .

Este rasgo se encuentra descrito en la Ontología del Fenotipo Humano (HPO) bajo el identificador HP:0002344 – Progressive neurologic deterioration, que agrupa un conjunto heterogéneo de enfermedades hereditarias y adquiridas, entre ellas ataxias, encefalopatías metabólicas y demencias frontotemporales \cite{Wakap2020} . Aunque cada una de estas patologías es ultrarrara (frecuencia <1:1.000.000), su frecuencia combinada alcanza aproximadamente 1 por cada 100.000 personas, lo que convierte a este fenotipo en un problema relevante dentro de las enfermedades neurológicas raras \cite{HPO_term_HP0002344} .

La HPO constituye una herramienta clave para vincular fenotipos clínicos con genes causales mediante un vocabulario jerárquico estandarizado \cite{Wakap2019} . Esta ontología ha mejorado la interpretación de variantes genéticas y ha facilitado el diagnóstico computacional de enfermedades raras basadas en descripciones fenotípicas precisas. 
Sin embargo, la heterogeneidad genética y fenotípica de las enfermedades asociadas a HP:0002344 dificulta la identificación de mecanismos comunes de patogénesis. Para acotar el análisis, nos centraremos en un conjunto representativo de cuatro patologías con este fenotipo: Spinocerebellar ataxia with epilepsy (ORPHA:778), Leigh syndrome (ORPHA:506), Frontotemporal lobar degeneration with TDP-43 inclusions (OMIM:607485) y Rett syndrome (ORPHA:778). Estas patologías ejemplifican distintos mecanismos de neurodegeneración —epigenético, metabólico y proteopático—, permitiendo explorar posibles rutas convergentes en el deterioro neurológico progresivo.

En la ataxia espinocerebelosa con epilepsia y el síndrome de Rett, las mutaciones en MECP2 alteran la regulación epigenética y la expresión génica neuronal. MECP2 codifica una proteína que se une al ADN metilado y modula la transcripción de genes esenciales para la plasticidad sináptica y la maduración neuronal; su disfunción produce desequilibrios en la excitabilidad cortical y pérdida progresiva de funciones cognitivas y motoras.

La enfermedad de Leigh, asociada a mutaciones en IARS2, afecta la función de la isoleucil-ARNt sintetasa mitocondrial, lo que compromete la traducción proteica dentro de la mitocondria y conduce a un déficit energético severo, estrés oxidativo y degeneración neuronal en regiones con alta demanda metabólica.
Por su parte, la degeneración lobar frontotemporal con inclusiones de TDP-43, causada por mutaciones en GRN, se caracteriza por la pérdida de progranulina, una proteína implicada en la homeostasis lisosomal, la inflamación y la supervivencia neuronal. La deficiencia de progranulina conduce a la acumulación de inclusiones citoplasmáticas de TDP-43, disfunción sináptica y neuroinflamación crónica.

A partir de esta base, se plantea la hipótesis de que la degeneración neurológica progresiva surge como resultado de alteraciones en redes moleculares interconectadas, compartidas por distintas enfermedades raras con orígenes genéticos diversos. 

La integración de información procedente de bases de datos como HPO, OMIM, STRINGdb y GeneCards permitirá construir una visión sistémica del fenotipo HP:0002344, identificando redes funcionales compartidas, genes candidatos no descritos y nodos terapéuticos potenciales. Este enfoque de biología de sistemas busca arrojar luz a los mecanismos moleculares de convergencia que subyacen al deterioro neurológico progresivo y aportar una base para el desarrollo de estrategias terapéuticas dirigidas \cite{HPO2025,Ehrhart2016_Rett_MECP2_pathways,Dong2024_IARS2_Leigh} .
	\section{Objetivos}
El presente trabajo tiene como finalidad estudiar los mecanismos moleculares implicados en el fenotipo \textbf{HP:0002344 – Progressive neurologic deterioration} desde un enfoque de biología de sistemas. Para ello, se plantean los siguientes objetivos específicos:

	
\begin{enumerate}
	
		\item \textbf{Identificar} los genes y las enfermedades asociadas al fenotipo HP:0002344 a partir de bases de datos como \textit{HPO}, \textit{OMIM} y \textit{GeneCards}.
		
		\item \textbf{Describir} los mecanismos moleculares implicados en las enfermedades representativas —Spinocerebellar ataxia with epilepsy, Leigh syndrome, Frontotemporal lobar degeneration y Rett syndrome— y sus genes causales (\textit{MECP2}, \textit{IARS2} y \textit{GRN}).
		
		\item \textbf{Construir y analizar} redes de interacción proteína-proteína (\textit{PPI}) a partir de los genes identificados mediante la base de datos \textit{STRINGdb}, evaluando métricas topológicas como el grado, la centralidad y la modularidad.
		
		\item \textbf{Detectar} grupos funcionales de genes y realizar un análisis de enriquecimiento para identificar procesos biológicos y rutas moleculares comunes.
		
		\item \textbf{Desarrollar} y documentar el código necesario para reproducir el análisis completo y garantizar la trazabilidad de los resultados en el repositorio de \textit{GitHub}.
	\end{enumerate}

	\input{tex_files/material_methods.tex}
	\input{tex_files/resultados.tex}
	\input{tex_files/discusion.tex}
	\input{tex_files/conclusiones.tex}
	
	
	%%%%%%%%%%%%%%%%%%%%%%%%%%%%%%%%%%%%%%%%%%%%%%
	%% OTRA INFORMACIÓN                         %%
	%%%%%%%%%%%%%%%%%%%%%%%%%%%%%%%%%%%%%%%%%%%%%%
	
	\begin{backmatter}
	
		\section*{Abreviaciones}%% if any
			Indicar lista de abreviaciones mostrando cada acrónimo a que corresponde
		
		\section*{Disponibilidad de datos y materiales}%% if any
			Debéis indicar aquí un enlace a vuestro repositorio de github.
		
		\section*{Contribución de los autores}
			Usando las iniciales que habéis definido al comienzo del documento, debeis indicar la contribución al proyecto en el estilo:
			J.E : Encargado del análisis de coexpresión con R, escritura de resultados; J.R.S : modelado de red con python y automatizado del código, escritura de métodos; ...
			OJO: que sea realista con los registros que hay en vuestros repositorios de github. 
		
		
		%%%%%%%%%%%%%%%%%%%%%%%%%%%%%%%%%%%%%%%%%%%%%%%%%%%%%%%%%%%%%%%%%%%%%%%%%%%%%%%%%%%%%%%%
		%% BIBLIOGRAFIA: no teneis que tocar nada, solo sustituir el archivo bibliography.bib %%
		%% por el que hayais generado vosotros                                                %%
		%%%%%%%%%%%%%%%%%%%%%%%%%%%%%%%%%%%%%%%%%%%%%%%%%%%%%%%%%%%%%%%%%%%%%%%%%%%%%%%%%%%%%%%%
		
		\bibliographystyle{bmc-mathphys} % Style BST file (bmc-mathphys, vancouver, spbasic).
		\bibliography{bibliography}      % Bibliography file (usually '*.bib' )
	
	\end{backmatter}
\end{document}
