\documentclass{bmcart}

%%%%%%%%%%%%%%%%%%%%%%%%%%%%%%%%%%%%%%%%%%%%%%
%%                                          %%
%% CARGA DE PAQUETES DE LATEX               %%
%%                                          %%
%%%%%%%%%%%%%%%%%%%%%%%%%%%%%%%%%%%%%%%%%%%%%%

%%% Load packages
\usepackage{amsthm,amsmath}
\usepackage{graphicx}
%\RequirePackage[numbers]{natbib}
%\RequirePackage{hyperref}
\usepackage[utf8]{inputenc} %unicode support
%\usepackage[applemac]{inputenc} %applemac support if unicode package fails
%\usepackage[latin1]{inputenc} %UNIX support if unicode package fails


%%%%%%%%%%%%%%%%%%%%%%%%%%%%%%%%%%%%%%%%%%%%%%
%%                                          %%
%% COMIENZO DEL DOCUMENTO                   %%
%%                                          %%
%%%%%%%%%%%%%%%%%%%%%%%%%%%%%%%%%%%%%%%%%%%%%%

\begin{document}

	\begin{frontmatter}
	
		\begin{fmbox}
			\dochead{Research}
			
			%%%%%%%%%%%%%%%%%%%%%%%%%%%%%%%%%%%%%%%%%%%%%%
			%% INTRODUCIR TITULO PROYECTO               %%
			%%%%%%%%%%%%%%%%%%%%%%%%%%%%%%%%%%%%%%%%%%%%%%
			
			\title{Análisis de redes génicas asociadas al fenotipo HP:0002344 – Progressive neurologic deterioration mediante biología de sistemas}
			
			%%%%%%%%%%%%%%%%%%%%%%%%%%%%%%%%%%%%%%%%%%%%%%
			%% AUTORES. METER UNA ENTRADA AUTHOR        %%
			%% POR PERSONA                              %%
			%%%%%%%%%%%%%%%%%%%%%%%%%%%%%%%%%%%%%%%%%%%%%%
			
			\author[
			  addressref={aff1},                   % ESTA LINEA SE COPIA IGUAL PARA CADA AUTOR
			  corref={aff1},                       % ESTA LINEA SOLO DEBE TENERLA EL COORDINADOR DEL GRUPO
			  email={capi13@uma.es}   % VUESTRO CORREO ACTIVO
			]{\inits{P.R.L}\fnm{Patricia} \snm{Rodríguez Lidueña}} % inits: INICIALES DE AUTOR, fnm: NOMBRE DE AUTOR, snm: APELLIDOS DE AUTOR
			\author[
			  addressref={aff1},
			  email={achrafousti@uma.es}
			]{\inits{A.O.E}\fnm{Achraf} \snm{Ousti El Moussati}}
			\author[
			  addressref={aff1},
		 	  email={aissaomar@uma.es}
			]{\inits{A.O.E.C}\fnm{Aissa Omar} \snm{El Hammouti Chachoui}}
			\author[
			  addressref={aff1},
			  email={hugosalascalderon@uma.es}
			]{\inits{H.S.C}\fnm{Hugo} \snm{Salas Calderón}}
			
			
			
			
			%%%%%%%%%%%%%%%%%%%%%%%%%%%%%%%%%%%%%%%%%%%%%%
			%% AFILIACION. NO TOCAR                     %%
			%%%%%%%%%%%%%%%%%%%%%%%%%%%%%%%%%%%%%%%%%%%%%%
			
			\address[id=aff1]{%                           % unique id
			  \orgdiv{ETSI Informática},             % department, if any
			  \orgname{Universidad de Málaga},          % university, etc
			  \city{Málaga},                              % city
			  \cny{España}                                    % country
			}
		
		\end{fmbox}% comment this for two column layout
		
		\begin{abstractbox}
		
			\begin{abstract} % abstract
			
			%%%%%%%%%%%%%%%%%%%%%%%%%%%%%%%%%%%%%%%%%%%%%%%
			%% RESUMEN BREVE DE NO MAS DE 100 PALABRAS   %%
			%%%%%%%%%%%%%%%%%%%%%%%%%%%%%%%%%%%%%%%%%%%%%%%	
			
			Las enfermedades neurodegenerativas tienen en común el fenotipo HP:0002344 – Progressive neurologic deterioration, caracterizado por la pérdida progresiva de funciones neuronales. Su heterogeneidad genética dificulta la identificación de mecanismos comunes, por lo que la biología de sistemas ofrece un marco integrador para su estudio. El presente trabajo analiza cuatro patologías representativas: Spinocerebellar ataxia with epilepsy, Leigh syndrome, Frontotemporal lobar degeneration with TDP-43 inclusions y Rett syndrome, asociadas a los genes MECP2, IARS2 y GRN. Se propone integrar información de bases de datos biomédicas y construir redes de interacción proteína–proteína (PPI) para identificar rutas moleculares compartidas, genes candidatos y posibles dianas terapéuticas implicadas en la neurodegeneración progresiva.
			
			\end{abstract}
			
			%%%%%%%%%%%%%%%%%%%%%%%%%%%%%%%%%%%%%%%%%%%%%%
			%% PALABRAS CLAVE DEL PROYECTO              %%
			%%%%%%%%%%%%%%%%%%%%%%%%%%%%%%%%%%%%%%%%%%%%%%
			
			\begin{keyword}
			\kwd{sample}
			\kwd{article}
			\kwd{author}
			\end{keyword}
		
		
		\end{abstractbox}
	
	\end{frontmatter}
	
	
	
	%%%%%%%%%%%%%%%%%%%%%%%%%%%%%%%%%
	%% COMIENZO DEL DOCUMENTO REAL %%
	%%%%%%%%%%%%%%%%%%%%%%%%%%%%%%%%%
	
	
\section{Introducción}

Las enfermedades neurodegenerativas constituyen uno de los mayores desafíos biomédicos contemporáneos debido a su alta prevalencia y a la ausencia de tratamientos curativos eficaces \cite{ref1}. Aunque sus manifestaciones clínicas son diversas, comparten un rasgo común: la degeneración neurológica progresiva, entendida como un deterioro gradual e irreversible de las funciones neuronales que afecta al movimiento, la cognición y la función sensorial \cite{ref2}.

Este rasgo se encuentra descrito en la Ontología del Fenotipo Humano (HPO) bajo el identificador HP:0002344 – Progressive neurologic deterioration, que agrupa un conjunto heterogéneo de enfermedades hereditarias y adquiridas, entre ellas ataxias, encefalopatías metabólicas y demencias frontotemporales \cite{ref3}. Aunque cada una de estas patologías es ultrarrara (frecuencia <1:1.000.000), su frecuencia combinada alcanza aproximadamente 1 por cada 100.000 personas, lo que convierte a este fenotipo en un problema relevante dentro de las enfermedades neurológicas raras \cite{ref4}.

La HPO constituye una herramienta clave para vincular fenotipos clínicos con genes causales mediante un vocabulario jerárquico estandarizado \cite{ref5}. Esta ontología ha mejorado la interpretación de variantes genéticas y ha facilitado el diagnóstico computacional de enfermedades raras basadas en descripciones fenotípicas precisas. 
Sin embargo, la heterogeneidad genética y fenotípica de las enfermedades asociadas a HP:0002344 dificulta la identificación de mecanismos comunes de patogénesis. Para acotar el análisis, nos centraremos en un conjunto representativo de cuatro patologías con este fenotipo: Spinocerebellar ataxia with epilepsy (ORPHA:778), Leigh syndrome (ORPHA:506), Frontotemporal lobar degeneration with TDP-43 inclusions (OMIM:607485) y Rett syndrome (ORPHA:778). Estas patologías ejemplifican distintos mecanismos de neurodegeneración —epigenético, metabólico y proteopático—, permitiendo explorar posibles rutas convergentes en el deterioro neurológico progresivo.

En la ataxia espinocerebelosa con epilepsia y el síndrome de Rett, las mutaciones en MECP2 alteran la regulación epigenética y la expresión génica neuronal. MECP2 codifica una proteína que se une al ADN metilado y modula la transcripción de genes esenciales para la plasticidad sináptica y la maduración neuronal; su disfunción produce desequilibrios en la excitabilidad cortical y pérdida progresiva de funciones cognitivas y motoras.

La enfermedad de Leigh, asociada a mutaciones en IARS2, afecta la función de la isoleucil-ARNt sintetasa mitocondrial, lo que compromete la traducción proteica dentro de la mitocondria y conduce a un déficit energético severo, estrés oxidativo y degeneración neuronal en regiones con alta demanda metabólica.
Por su parte, la degeneración lobar frontotemporal con inclusiones de TDP-43, causada por mutaciones en GRN, se caracteriza por la pérdida de progranulina, una proteína implicada en la homeostasis lisosomal, la inflamación y la supervivencia neuronal. La deficiencia de progranulina conduce a la acumulación de inclusiones citoplasmáticas de TDP-43, disfunción sináptica y neuroinflamación crónica.

A partir de esta base, se plantea la hipótesis de que la degeneración neurológica progresiva surge como resultado de alteraciones en redes moleculares interconectadas, compartidas por distintas enfermedades raras con orígenes genéticos diversos. 

La integración de información procedente de bases de datos como HPO, OMIM, STRINGdb y GeneCards permitirá construir una visión sistémica del fenotipo HP:0002344, identificando redes funcionales compartidas, genes candidatos no descritos y nodos terapéuticos potenciales. Este enfoque de biología de sistemas busca arrojar luz a los mecanismos moleculares de convergencia que subyacen al deterioro neurológico progresivo y aportar una base para el desarrollo de estrategias terapéuticas dirigidas \cite{ref6,ref7,ref8}.
	\section{Objetivos}
El presente trabajo tiene como finalidad estudiar los mecanismos moleculares implicados en el fenotipo \textbf{HP:0002344 – Progressive neurologic deterioration} desde un enfoque de biología de sistemas. Para ello, se plantean los siguientes objetivos específicos:

	
\begin{enumerate}
	
		\item \textbf{Identificar} los genes y las enfermedades asociadas al fenotipo HP:0002344 a partir de bases de datos como \textit{HPO}, \textit{OMIM} y \textit{GeneCards}.
		
		\item \textbf{Describir} los mecanismos moleculares implicados en las enfermedades representativas —Spinocerebellar ataxia with epilepsy, Leigh syndrome, Frontotemporal lobar degeneration y Rett syndrome— y sus genes causales (\textit{MECP2}, \textit{IARS2} y \textit{GRN}).
		
		\item \textbf{Construir y analizar} redes de interacción proteína-proteína (\textit{PPI}) a partir de los genes identificados mediante la base de datos \textit{STRINGdb}, evaluando métricas topológicas como el grado, la centralidad y la modularidad.
		
		\item \textbf{Detectar} grupos funcionales de genes y realizar un análisis de enriquecimiento para identificar procesos biológicos y rutas moleculares comunes.
		
		\item \textbf{Desarrollar} y documentar el código necesario para reproducir el análisis completo y garantizar la trazabilidad de los resultados en el repositorio de \textit{GitHub}.
	\end{enumerate}

	\section{Materiales y Métodos}

\subsection{Materiales}

El análisis se basó en información procedente de bases de datos biomédicas de referencia.
Se seleccionó el término \textbf{HP:0002344 – Progressive neurologic deterioration} de la Human Phenotype Ontology (HPO) \cite{HPO2025} como punto de partida, recopilando los genes asociados a este fenotipo.
La relación gen–enfermedad se contrastó con OMIM \cite{Amberger2011} y GeneCards \cite{GeneCards}, lo que permitió confirmar y ampliar la evidencia disponible sobre cada gen.

Las interacciones entre las proteínas codificadas por dichos genes se obtuvieron mediante STRINGdb (versión 12.0) \cite{STRING}, considerando únicamente interacciones de alta confianza (combined score $\geq$ 0.7) en \textit{Homo sapiens}.

El procesamiento y análisis de redes se realizaron en R (versión 4.3.2), empleando los paquetes \texttt{igraph}, \texttt{tidygraph}, \texttt{ggraph} y \texttt{linkcomm} 
\cite{igraph,tidygraph,ggraph,Kalinka2011_linkcomm}.
Todos los scripts, tablas y visualizaciones se encuentran disponibles en el repositorio de GitHub del proyecto, garantizando la reproducibilidad del estudio.

\subsection{Metodología}

\subsubsection{Flujo de trabajo general}

El procedimiento seguido en el estudio puede resumirse en las siguientes etapas:

\begin{enumerate}
	\item Selección del fenotipo HP:0002344 en HPO.
	\item Extracción de genes asociados al fenotipo.
	\item Validación de las asociaciones gen–enfermedad mediante OMIM y GeneCards.
	\item Obtención de interacciones proteína–proteína (PPI) en STRINGdb.
	\item Construcción y depuración de la red en R.
	\item Cálculo de métricas estructurales de la red.
	\item Identificación de comunidades mediante Louvain, Walktrap y linkcomm.
	\item Análisis detallado de los genes de interés: \textit{MECP2}, \textit{IARS2} y \textit{GRN}.
	\item Enriquecimiento funcional de los módulos identificados (GO y KEGG).
	\item Interpretación integrada de resultados.
\end{enumerate}

\subsubsection{Recopilación de genes y construcción de la red}

A partir del término HP:0002344 se descargó la lista de genes relacionados desde HPO \cite{HPO_term_HP0002344}.
Tras eliminar duplicados, se verificó la validez de las asociaciones gen–enfermedad mediante OMIM \cite{Amberger2011} y GeneCards \cite{GeneCards}, obteniendo un conjunto depurado de genes candidatos.

Estos genes se introdujeron en STRINGdb (v12.0) \cite{STRING} para obtener la red de interacciones proteína–proteína (PPI), restringiéndonos a interacciones con evidencia experimental o provenientes de bases de datos curadas.
La red se exportó y procesó en R, transformándose en un objeto \texttt{igraph} para su análisis estructural.

\subsubsection{Análisis estructural de la red}

Se caracterizó la topología global mediante métricas estándar de redes.
El \textit{grado} permitió identificar genes altamente conectados (posibles hubs), mientras que las centralidades de intermediación y cercanía ayudaron a detectar nodos clave en la comunicación entre módulos.

Se calculó el coeficiente de agrupamiento para evaluar la tendencia de los nodos a formar clústeres, y la modularidad global para estimar el grado de organización de la red en comunidades funcionales.

\subsubsection{Identificación de comunidades funcionales}

La detección de comunidades se realizó mediante dos algoritmos ampliamente utilizados: Louvain \cite{Louvain} y Walktrap \cite{Walktrap}, implementados en \texttt{igraph}.
Además, se aplicó el enfoque de comunidades solapadas mediante el paquete \texttt{linkcomm} \cite{Kalinka2011_linkcomm}, permitiendo que un mismo gen pudiera participar en distintos módulos cuando intervenía en procesos independientes.

Cada comunidad se evaluó considerando su cohesión, su tamaño y la presencia de genes previamente relacionados con enfermedades neurológicas.

\subsubsection{Análisis de genes de interés}

Se estudiaron en mayor profundidad los genes \textbf{MECP2}, \textbf{IARS2} y \textbf{GRN}, seleccionados por su relevancia clínica en patologías asociadas al deterioro neurológico progresivo.

El gen \textit{MECP2} está implicado en el síndrome de Rett \cite{Ehrhart2016_Rett_MECP2_pathways} y en el síndrome por duplicación MECP2 \cite{dePalma2012_eating_spasms_MECP2_dup}.
El gen \textit{IARS2} se ha asociado a síndrome de Leigh con deficiencia combinada en fosforilación oxidativa \cite{Dong2024_IARS2_Leigh}.
Por último, \textit{GRN} está implicado en la degeneración lobar frontotemporal \cite{Karamysheva2019_Granulin_FTLD}.

Para cada gen se generó una subred local con sus interacciones más próximas, visualizada mediante \texttt{ggraph}, destacando su conectividad y su relación con nodos relevantes dentro de la red global.

\subsubsection{Enriquecimiento funcional}

Finalmente, las comunidades más representativas se sometieron a análisis de enriquecimiento funcional mediante las herramientas integradas en STRINGdb.
Se evaluaron las categorías de Gene Ontology (procesos biológicos, funciones moleculares y componentes celulares) y las rutas KEGG, considerando significativos los términos con p ajustado < 0.05 (corrección Benjamini–Hochberg).

Este análisis permitió relacionar los módulos detectados con procesos neuronales y rutas metabólicas asociadas al deterioro neurológico progresivo, en consonancia con los mecanismos descritos en la literatura biomédica \cite{Kelser2024,Gao2008}.

	\section{Resultados}

\subsection{Red PPI asociada al fenotipo HP:0002344}

La consulta al término HP:0002344 en la Human Phenotype Ontology produjo un conjunto inicial de \textbf{XX genes} asociados al fenotipo. Tras la depuración de duplicados y el mapeo en STRINGdb, \textbf{XX genes} fueron identificados correctamente y empleados para la construcción de la red PPI.

La red final incluyó un total de \textbf{XX nodos} y \textbf{XX interacciones} con un umbral de confianza de 0.7 (combined score $\geq$ 700). La densidad obtenida fue de \textbf{X.XXX}, con un grado medio de \textbf{X.X}. La Tabla~\ref{tab:metricas_globales} resume las métricas topológicas principales de la red.

%\begin{figure}[H]
%	\centering
%	\includegraphics[width=0.85\linewidth]{FIGURA_PPI.png}
%	\caption{Red PPI construida a partir de los genes asociados a HP:0002344 tras el filtrado de interacciones de alta confianza. Los nodos representan genes y las aristas interacciones proteína--proteína.}
%	\label{fig:ppi}
%\end{figure}
%
%\begin{table}[H]
%	\centering
%	\caption{Métricas topológicas globales de la red PPI.}
%	\label{tab:metricas_globales}
%	\begin{tabular}{l c}
%		\hline
%		Métrica & Valor \\
%		\hline
%		Número de nodos & XX \\
%		Número de aristas & XX \\
%		Densidad & X.XXX \\
%		Grado medio & X.X \\
%		Coeficiente de agrupamiento global & X.XXX \\
%		Moduralidad (Louvain) & X.XXX \\
%		\hline
%	\end{tabular}
%\end{table}

\subsection{Distribución de grados y nodos altamente conectados}

La distribución de grados mostró valores comprendidos entre \textbf{X} y \textbf{X}. El nodo con mayor conectividad presentó un grado de \textbf{X}. El histograma de grados obtenido se presenta en la Figura~\ref{fig:histograma_grado}.

%\begin{figure}[H]
%	\centering
%	\includegraphics[width=0.75\linewidth]{FIGURA_HISTOGRAMA_GRADOS.png}
%	\caption{Distribución de grados de la red PPI. El eje X representa el grado y el eje Y la frecuencia.}
%	\label{fig:histograma_grado}
%\end{figure}

\subsection{Detección de comunidades}

La aplicación del algoritmo Louvain detectó un total de \textbf{X comunidades}, con un valor de modularidad de \textbf{X.XXX}. Por su parte, el algoritmo Walktrap identificó \textbf{X comunidades}, alcanzando una modularidad de \textbf{X.XXX}. La Figura~\ref{fig:louvain} muestra la partición obtenida mediante Louvain y la Figura~\ref{fig:walktrap} representa la partición correspondiente a Walktrap.

%\begin{figure}[H]
%	\centering
%	\includegraphics[width=0.8\linewidth]{FIGURA_LOUVAIN.png}
%	\caption{Partición comunitaria generada mediante el algoritmo Louvain. Los colores representan comunidades distintas.}
%	\label{fig:louvain}
%\end{figure}
%
%\begin{figure}[H]
%	\centering
%	\includegraphics[width=0.8\linewidth]{FIGURA_WALKTRAP.png}
%	\caption{Partición comunitaria generada mediante el algoritmo Walktrap. Los colores representan comunidades distintas.}
%	\label{fig:walktrap}
%\end{figure}

\subsection{Métricas individuales y genes de interés}

Las métricas de grado, centralidad de intermediación, cercanía y coeficiente de agrupamiento se calcularon para cada nodo de la red. La Tabla~\ref{tab:metricas_individuales} resume estos valores.

%\begin{table}[H]
%	\centering
%	\caption{Métricas topológicas individuales de los genes analizados.}
%	\label{tab:metricas_individuales}
%	\begin{tabular}{l c c c c}
%		\hline
%		Gen & Grado & Betweenness & Closeness & Clustering \\
%		\hline
%		MECP2 & X & X.XXX & X.XXX & X.XXX \\
%		IARS2 & X & X.XXX & X.XXX & X.XXX \\
%		GRN   & X & X.XXX & X.XXX & X.XXX \\
%		\hline
%	\end{tabular}
%\end{table}

Se generaron subredes de vecindad inmediata (orden 1) para los genes \textit{MECP2}, \textit{IARS2} y \textit{GRN}. La Figura~\ref{fig:ego_mecp2} muestra la subred de MECP2 y las Figuras~\ref{fig:ego_iars2} y~\ref{fig:ego_grn} las correspondientes a los otros genes.
%
%\begin{figure}[H]
%	\centering
%	\includegraphics[width=0.8\linewidth]{FIGURA_EGO_MECP2.png}
%	\caption{Subred de vecindad de orden 1 del gen \textit{MECP2}.}
%	\label{fig:ego_mecp2}
%\end{figure}

\subsection{Enriquecimiento funcional}

El análisis GO identificó \textbf{X términos significativamente enriquecidos} (p ajustado < 0.05). Los términos principales estuvieron relacionados con \textbf{XXX}, \textbf{XXX} y \textbf{XXX}. El análisis KEGG detectó \textbf{X rutas significativas} tras la conversión a identificadores Entrez.

Las Figuras~\ref{fig:go} y~\ref{fig:kegg} muestran los dotplots obtenidos para GO y KEGG, respectivamente.

%\begin{figure}[H]
%	\centering
%	\includegraphics[width=0.75\linewidth]{FIGURA_GO.png}
%	\caption{Dotplot del análisis de enriquecimiento GO (procesos biológicos).}
%	\label{fig:go}
%\end{figure}
%
%\begin{figure}[H]
%	\centering
%	\includegraphics[width=0.75\linewidth]{FIGURA_KEGG.png}
%	\caption{Dotplot del análisis KEGG.}
%	\label{fig:kegg}
%\end{figure}

\subsection{Resumen de los resultados}

En conjunto, el análisis de la red PPI, la detección de comunidades, la caracterización de métricas topológicas y los análisis de enriquecimiento funcional proporcionaron una visión estructural y funcional del conjunto de genes asociados al fenotipo HP:0002344. La integración de estos resultados se desarrolla en la sección de Discusión.


	\input{tex_files/discusion.tex}
	\input{tex_files/conclusiones.tex}
	
	
	%%%%%%%%%%%%%%%%%%%%%%%%%%%%%%%%%%%%%%%%%%%%%%
	%% OTRA INFORMACIÓN                         %%
	%%%%%%%%%%%%%%%%%%%%%%%%%%%%%%%%%%%%%%%%%%%%%%
	
	\begin{backmatter}
	
		\section*{Abreviaciones}%% if any
			Indicar lista de abreviaciones mostrando cada acrónimo a que corresponde
		
		\section*{Disponibilidad de datos y materiales}%% if any
			Debéis indicar aquí un enlace a vuestro repositorio de github.
		
		\section*{Contribución de los autores}
			Usando las iniciales que habéis definido al comienzo del documento, debeis indicar la contribución al proyecto en el estilo:
			J.E : Encargado del análisis de coexpresión con R, escritura de resultados; J.R.S : modelado de red con python y automatizado del código, escritura de métodos; ...
			OJO: que sea realista con los registros que hay en vuestros repositorios de github. 
		
		
		%%%%%%%%%%%%%%%%%%%%%%%%%%%%%%%%%%%%%%%%%%%%%%%%%%%%%%%%%%%%%%%%%%%%%%%%%%%%%%%%%%%%%%%%
		%% BIBLIOGRAFIA: no teneis que tocar nada, solo sustituir el archivo bibliography.bib %%
		%% por el que hayais generado vosotros                                                %%
		%%%%%%%%%%%%%%%%%%%%%%%%%%%%%%%%%%%%%%%%%%%%%%%%%%%%%%%%%%%%%%%%%%%%%%%%%%%%%%%%%%%%%%%%
		
		\bibliographystyle{bmc-mathphys} % Style BST file (bmc-mathphys, vancouver, spbasic).
		\bibliography{bibliography}      % Bibliography file (usually '*.bib' )
	
	\end{backmatter}
\end{document}
