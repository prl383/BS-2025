\section{Objetivos}
El presente trabajo tiene como finalidad estudiar los mecanismos moleculares implicados en el fenotipo \textbf{HP:0002344 – Progressive neurologic deterioration} desde un enfoque de biología de sistemas. Para ello, se plantean los siguientes objetivos específicos:

	
\begin{enumerate}
	
		\item \textbf{Identificar} los genes y las enfermedades asociadas al fenotipo HP:0002344 a partir de bases de datos como \textit{HPO}, \textit{OMIM} y \textit{GeneCards}.
		
		\item \textbf{Describir} los mecanismos moleculares implicados en las enfermedades representativas —Spinocerebellar ataxia with epilepsy, Leigh syndrome, Frontotemporal lobar degeneration y Rett syndrome— y sus genes causales (\textit{MECP2}, \textit{IARS2} y \textit{GRN}).
		
		\item \textbf{Construir y analizar} redes de interacción proteína-proteína (\textit{PPI}) a partir de los genes identificados mediante la base de datos \textit{STRINGdb}, evaluando métricas topológicas como el grado, la centralidad y la modularidad.
		
		\item \textbf{Detectar} grupos funcionales de genes y realizar un análisis de enriquecimiento para identificar procesos biológicos y rutas moleculares comunes.
		
		\item \textbf{Desarrollar} y documentar el código necesario para reproducir el análisis completo y garantizar la trazabilidad de los resultados en el repositorio de \textit{GitHub}.
	\end{enumerate}
