\section{Materiales y Métodos}

\subsection{Materiales}

El análisis se basó en información procedente de bases de datos biomédicas de referencia.
Se seleccionó el término \textbf{HP:0002344 – Progressive neurologic deterioration} de la Human Phenotype Ontology (HPO) \cite{HPO2025} como punto de partida, recopilando los genes asociados a este fenotipo.
La relación gen–enfermedad se contrastó con OMIM \cite{Amberger2011} y GeneCards \cite{GeneCards}, lo que permitió confirmar y ampliar la evidencia disponible sobre cada gen.

Las interacciones entre las proteínas codificadas por dichos genes se obtuvieron mediante STRINGdb (versión 12.0) \cite{STRING}, considerando únicamente interacciones de alta confianza (combined score $\geq$ 0.7) en \textit{Homo sapiens}.

El procesamiento y análisis de redes se realizaron en R (versión 4.3.2), empleando los paquetes \texttt{igraph}, \texttt{tidygraph}, \texttt{ggraph} y \texttt{linkcomm} 
\cite{igraph,tidygraph,ggraph,Kalinka2011_linkcomm}.
Todos los scripts, tablas y visualizaciones se encuentran disponibles en el repositorio de GitHub del proyecto, garantizando la reproducibilidad del estudio.

\subsection{Metodología}

\subsubsection{Flujo de trabajo general}

El procedimiento seguido en el estudio puede resumirse en las siguientes etapas:

\begin{enumerate}
	\item Selección del fenotipo HP:0002344 en HPO.
	\item Extracción de genes asociados al fenotipo.
	\item Validación de las asociaciones gen–enfermedad mediante OMIM y GeneCards.
	\item Obtención de interacciones proteína–proteína (PPI) en STRINGdb.
	\item Construcción y depuración de la red en R.
	\item Cálculo de métricas estructurales de la red.
	\item Identificación de comunidades mediante Louvain, Walktrap y linkcomm.
	\item Análisis detallado de los genes de interés: \textit{MECP2}, \textit{IARS2} y \textit{GRN}.
	\item Enriquecimiento funcional de los módulos identificados (GO y KEGG).
	\item Interpretación integrada de resultados.
\end{enumerate}

\subsubsection{Recopilación de genes y construcción de la red}

A partir del término HP:0002344 se descargó la lista de genes relacionados desde HPO \cite{HPO_term_HP0002344}.
Tras eliminar duplicados, se verificó la validez de las asociaciones gen–enfermedad mediante OMIM \cite{Amberger2011} y GeneCards \cite{GeneCards}, obteniendo un conjunto depurado de genes candidatos.

Estos genes se introdujeron en STRINGdb (v12.0) \cite{STRING} para obtener la red de interacciones proteína–proteína (PPI), restringiéndonos a interacciones con evidencia experimental o provenientes de bases de datos curadas.
La red se exportó y procesó en R, transformándose en un objeto \texttt{igraph} para su análisis estructural.

\subsubsection{Análisis estructural de la red}

Se caracterizó la topología global mediante métricas estándar de redes.
El \textit{grado} permitió identificar genes altamente conectados (posibles hubs), mientras que las centralidades de intermediación y cercanía ayudaron a detectar nodos clave en la comunicación entre módulos.

Se calculó el coeficiente de agrupamiento para evaluar la tendencia de los nodos a formar clústeres, y la modularidad global para estimar el grado de organización de la red en comunidades funcionales.

\subsubsection{Identificación de comunidades funcionales}

La detección de comunidades se realizó mediante dos algoritmos ampliamente utilizados: Louvain \cite{Louvain} y Walktrap \cite{Walktrap}, implementados en \texttt{igraph}.
Además, se aplicó el enfoque de comunidades solapadas mediante el paquete \texttt{linkcomm} \cite{Kalinka2011_linkcomm}, permitiendo que un mismo gen pudiera participar en distintos módulos cuando intervenía en procesos independientes.

Cada comunidad se evaluó considerando su cohesión, su tamaño y la presencia de genes previamente relacionados con enfermedades neurológicas.

\subsubsection{Análisis de genes de interés}

Se estudiaron en mayor profundidad los genes \textbf{MECP2}, \textbf{IARS2} y \textbf{GRN}, seleccionados por su relevancia clínica en patologías asociadas al deterioro neurológico progresivo.

El gen \textit{MECP2} está implicado en el síndrome de Rett \cite{Ehrhart2016_Rett_MECP2_pathways} y en el síndrome por duplicación MECP2 \cite{dePalma2012_eating_spasms_MECP2_dup}.
El gen \textit{IARS2} se ha asociado a síndrome de Leigh con deficiencia combinada en fosforilación oxidativa \cite{Dong2024_IARS2_Leigh}.
Por último, \textit{GRN} está implicado en la degeneración lobar frontotemporal \cite{Karamysheva2019_Granulin_FTLD}.

Para cada gen se generó una subred local con sus interacciones más próximas, visualizada mediante \texttt{ggraph}, destacando su conectividad y su relación con nodos relevantes dentro de la red global.

\subsubsection{Enriquecimiento funcional}

Finalmente, las comunidades más representativas se sometieron a análisis de enriquecimiento funcional mediante las herramientas integradas en STRINGdb.
Se evaluaron las categorías de Gene Ontology (procesos biológicos, funciones moleculares y componentes celulares) y las rutas KEGG, considerando significativos los términos con p ajustado < 0.05 (corrección Benjamini–Hochberg).

Este análisis permitió relacionar los módulos detectados con procesos neuronales y rutas metabólicas asociadas al deterioro neurológico progresivo, en consonancia con los mecanismos descritos en la literatura biomédica \cite{Kelser2024,Gao2008}.
