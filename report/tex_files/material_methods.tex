\section{Materiales y métodos}
\textbf{4. Materiales y Métodos}
\textbf{4.1 Materiales}

El análisis se basó en información procedente de bases de datos biomédicas de referencia.
Se seleccionó el término \textbf{HP:0002344 – Progressive neurologic deterioration de la Human Phenotype Ontology (HPO)} como punto de partida, recopilando los genes asociados a este fenotipo.
La información genética se contrastó con OMIM y GeneCards para confirmar la relación gen-enfermedad y ampliar la evidencia funcional.
Las interacciones entre las proteínas codificadas por dichos genes se obtuvieron mediante STRINGdb (versión 12.0), considerando únicamente interacciones de alta confianza (combined score mayor o igual a 0.7) en el organismo Homo sapiens.

El procesamiento y análisis de redes se llevaron a cabo en R (versión 4.3.2), utilizando los paquetes igraph, tidygraph, ggraph y linkcomm.
Todas las rutinas y resultados se encuentran disponibles en el repositorio de GitHub del proyecto para garantizar la reproducibilidad del estudio.

\textbf{4.2 Metodología
\textbf{4.2.1 Recopilación de genes y construcción de la red}}

A partir del término HP:0002344 se descargó la lista de genes relacionados con el fenotipo desde HPO.
Tras eliminar duplicados y verificar la validez de las asociaciones mediante OMIM y GeneCards, se generó una lista depurada de genes candidatos.
Estos genes se introdujeron en STRINGdb para obtener la red de interacciones proteína–proteína (PPI), incluyendo únicamente conexiones respaldadas por evidencia experimental o de bases de datos consolidadas.
La red se exportó y procesó en R, donde se transformó en un objeto igraph para su análisis estructural.

\textbf{4.2.2 Análisis estructural de la red}

Se caracterizó la topología de la red mediante medidas de conectividad y centralidad.
El grado permitió identificar genes con un gran número de interacciones (posibles “hubs”), mientras que las medidas de centralidad de intermediación y cercanía ayudaron a localizar genes con un papel relevante en la comunicación entre módulos.
También se calculó el coeficiente de agrupamiento, que refleja la tendencia de los genes a formar clústeres, y la modularidad global, que evalúa el grado de organización de la red en comunidades funcionales.
Estas métricas proporcionaron una visión general del nivel de interconexión y jerarquía del sistema.

\textbf{4.2.3 Identificación de comunidades funcionales}

Para detectar grupos de genes con funciones relacionadas se aplicaron los algoritmos Louvain y Walktrap, disponibles en igraph.
Ambos métodos agrupan nodos con alta densidad de conexiones internas, lo que facilita la detección de posibles módulos biológicos.
Adicionalmente, se empleó el enfoque de comunidades solapadas mediante el paquete linkcomm, permitiendo que un mismo gen participara en varios módulos cuando intervenía en diferentes procesos biológicos.
Cada comunidad se evaluó según su tamaño, cohesión y la presencia de genes asociados a enfermedades relevantes.

\textbf{4.2.4 Análisis de genes de interés}

Se examinaron de forma específica los genes \textbf{MECP2}, \textbf{IARS2} y \textbf{GRN}, seleccionados por su implicación directa en enfermedades representativas del fenotipo.
Para cada uno se generó una subred con sus interacciones más próximas, destacando su conectividad y la relación con otros nodos relevantes dentro de la red global.
Las visualizaciones se realizaron con ggraph, resaltando los módulos a los que pertenecían y su posible papel como elementos reguladores.

\textbf{4.2.5 Enriquecimiento funcional}

Las comunidades más relevantes se sometieron a análisis de enriquecimiento funcional mediante las herramientas integradas en STRINGdb.
Se evaluaron las categorías de Gene Ontology (procesos biológicos, funciones moleculares y componentes celulares) y las rutas KEGG.
Los términos se consideraron significativos cuando presentaron un valor de p ajustado menor que \textbf{0.05} (corrección de Benjamini–Hochberg).
Este análisis permitió relacionar los módulos detectados con procesos neuronales y rutas metabólicas asociadas al deterioro neurológico progresivo.