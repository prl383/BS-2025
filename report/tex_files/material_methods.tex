\section{Materiales y Métodos}

\subsection{Materiales}

El análisis se basó en información procedente de bases de datos biomédicas de referencia.
Se seleccionó el término \textbf{HP:0002344 – Progressive neurologic deterioration} de la Human Phenotype Ontology (HPO) \cite{HPO2025} como punto de partida, recopilando los genes asociados a este fenotipo.
La relación gen--enfermedad se contrastó con OMIM \cite{Amberger2011} y GeneCards \cite{GeneCards}, lo que permitió confirmar y ampliar la evidencia disponible sobre cada gen.

Las interacciones entre las proteínas codificadas por dichos genes se obtuvieron mediante STRINGdb (versión 12.0) \cite{STRING}, considerando únicamente interacciones de alta confianza (combined score $\geq$ 0.7) en \textit{Homo sapiens}.

El procesamiento y análisis de redes se realizaron en R (versión 4.3.2), empleando los paquetes \texttt{igraph}, \texttt{tidygraph}, \texttt{ggraph} y \texttt{linkcomm} \cite{igraph,tidygraph,ggraph,Kalinka2011_linkcomm}.
Adicionalmente, se utilizaron los paquetes \texttt{clusterProfiler}, \texttt{enrichplot} y \texttt{org.Hs.eg.db} para el análisis de enriquecimiento funcional.
Todos los scripts, tablas y visualizaciones se encuentran disponibles en el repositorio de GitHub del proyecto, garantizando la reproducibilidad del estudio.

\subsection{Metodología}

\subsubsection{Flujo de trabajo general}

El procedimiento seguido en el estudio puede resumirse en las siguientes etapas:

\begin{enumerate}
	\item Selección del fenotipo HP:0002344 en HPO.
	\item Extracción de genes asociados al fenotipo mediante la API de HPO.
	\item Validación de las asociaciones gen--enfermedad mediante OMIM y GeneCards.
	\item Obtención de interacciones proteína--proteína (PPI) en STRINGdb.
	\item Construcción y depuración de la red en R.
	\item Cálculo de métricas estructurales de la red.
	\item Identificación de comunidades mediante Louvain y Walktrap.
	\item Análisis detallado de los genes de interés: \textit{MECP2}, \textit{IARS2} y \textit{GRN}.
	\item Enriquecimiento funcional de los módulos identificados (GO y KEGG).
	\item Interpretación integrada de resultados.
\end{enumerate}

\subsubsection{Recopilación de genes y construcción de la red}

A partir del término HP:0002344 se obtuvo programáticamente la lista de genes relacionados mediante la API REST de HPO \cite{HPO_term_HP0002344}.
La consulta se realizó mediante el endpoint \texttt{https://ontology.jax.org/api/network/annotation/HP:0002344}, utilizando el paquete \texttt{httr} de R para realizar las peticiones HTTP y el paquete \texttt{jsonlite} para parsear las respuestas JSON.

Tras eliminar duplicados y valores nulos, se verificó la validez de las asociaciones gen--enfermedad mediante OMIM \cite{Amberger2011} y GeneCards \cite{GeneCards}, obteniendo un conjunto depurado de genes candidatos.

Estos genes se introdujeron en STRINGdb (v12.0) \cite{STRING} mediante el paquete de Bioconductor \texttt{STRINGdb}, estableciendo los siguientes parámetros:

\begin{itemize}
	\item \textbf{Organismo:} \textit{Homo sapiens} (NCBI Taxonomy ID: 9606)
	\item \textbf{Umbral de confianza:} 0.7 (combined score $\geq$ 700), correspondiente a interacciones de alta confianza
	\item \textbf{Versión de la base de datos:} 12.0
\end{itemize}

Para asegurar la calidad de las interacciones, únicamente se consideraron aquellas con evidencia experimental o provenientes de bases de datos curadas.
El mapeo de los símbolos génicos a identificadores STRING se realizó mediante la función \texttt{map()} del paquete \texttt{STRINGdb}, eliminando los genes que no pudieron ser mapeados.

La red resultante se construyó obteniendo todas las interacciones entre los genes mapeados mediante la función \texttt{get\_interactions()}.
Los datos de interacción se transformaron en un objeto de clase \texttt{igraph} para su análisis estructural, estableciendo los nombres de genes como atributos de los vértices y los scores de confianza como pesos de las aristas.

\subsubsection{Análisis estructural de la red}

Se caracterizó la topología global de la red mediante métricas estándar implementadas en el paquete \texttt{igraph}.

El \textbf{grado} de cada nodo (número de conexiones) se calculó mediante la función \texttt{degree()}, permitiendo identificar genes altamente conectados (posibles hubs).
Las centralidades de \textbf{intermediación} (\textit{betweenness centrality}) y \textbf{cercanía} (\textit{closeness centrality}) se computaron mediante las funciones \texttt{betweenness()} y \texttt{closeness()} respectivamente, ayudando a detectar nodos clave en la comunicación entre módulos.

Se calculó el \textbf{coeficiente de agrupamiento} (\textit{clustering coefficient}) local para cada nodo mediante \texttt{transitivity(type = "local")} y el coeficiente de agrupamiento global mediante \texttt{transitivity(type = "global")}, evaluando la tendencia de los nodos a formar clústeres densos.

La \textbf{modularidad} global de la red se estimó mediante la función \texttt{modularity()}, cuantificando el grado de organización de la red en comunidades funcionales.
Valores de modularidad superiores a 0.3 se consideran indicativos de estructura comunitaria significativa.

Todas las métricas calculadas se consolidaron en una tabla única que incluye, para cada gen: símbolo génico, grado, betweenness, closeness, clustering coefficient y membresía comunitaria.

\subsubsection{Identificación de comunidades funcionales}

La detección de comunidades se realizó mediante dos algoritmos complementarios implementados en \texttt{igraph}:

\begin{itemize}
	\item \textbf{Algoritmo de Louvain} \cite{Louvain}: Método de optimización multinivel que maximiza la modularidad de la red mediante agrupación jerárquica. Se aplicó mediante la función \texttt{cluster\_louvain()}.
	
	\item \textbf{Algoritmo de Walktrap} \cite{Walktrap}: Método basado en caminatas aleatorias que identifica comunidades mediante la estructura de caminos cortos en la red. Se aplicó mediante \texttt{cluster\_walktrap()} con 4 pasos de caminata aleatoria.
\end{itemize}

Para cada partición resultante se calculó su modularidad y se analizó la distribución de tamaños de las comunidades.
Los resultados de ambos métodos se compararon para evaluar la robustez de la estructura comunitaria detectada.

Cada comunidad se evaluó considerando su cohesión interna, su tamaño y la presencia de genes previamente relacionados con enfermedades neurológicas progresivas en la literatura.

\subsubsection{Análisis de genes de interés}

Se estudiaron en mayor profundidad los genes \textbf{MECP2}, \textbf{IARS2} y \textbf{GRN}, seleccionados por su relevancia clínica en patologías asociadas al deterioro neurológico progresivo.

El gen \textit{MECP2} está implicado en el síndrome de Rett \cite{Ehrhart2016_Rett_MECP2_pathways} y en el síndrome por duplicación MECP2 \cite{dePalma2012_eating_spasms_MECP2_dup}.
El gen \textit{IARS2} se ha asociado a síndrome de Leigh con deficiencia combinada en fosforilación oxidativa \cite{Dong2024_IARS2_Leigh}.
Por último, \textit{GRN} está implicado en la degeneración lobar frontotemporal \cite{Karamysheva2019_Granulin_FTLD}.

Para cada gen se verificó su presencia en la red y se extrajeron sus métricas topológicas (grado, centralidad de intermediación, coeficiente de agrupamiento).

Se generaron subredes de vecindad inmediata para cada gen de interés mediante la función \texttt{make\_ego\_graph()}, incluyendo el gen focal y todos sus vecinos directos (orden 1).
Estas subredes locales se visualizaron mediante \texttt{ggraph}, destacando la conectividad del gen focal y su relación con nodos relevantes dentro de la red global.
Los nodos se colorearon según su pertenencia a comunidades y se dimensionaron proporcionalmente a su grado.

\subsubsection{Enriquecimiento funcional}

El análisis de enriquecimiento funcional se realizó mediante el paquete \texttt{clusterProfiler} de Bioconductor.

Se evaluaron las categorías de \textbf{Gene Ontology} (GO) utilizando la función \texttt{enrichGO()}, con los siguientes parámetros:

\begin{itemize}
	\item \textbf{Base de datos de anotación:} \texttt{org.Hs.eg.db}
	\item \textbf{Ontología:} Procesos biológicos (BP, \textit{Biological Process})
	\item \textbf{Tipo de identificador:} SYMBOL
	\item \textbf{Método de corrección:} Benjamini--Hochberg (BH)
	\item \textbf{Umbral de significancia:} p ajustado < 0.05, q-valor < 0.2
\end{itemize}

El análisis se realizó tanto para la red completa como para cada comunidad individual que contenía al menos 5 genes.

Para el análisis de rutas metabólicas \textbf{KEGG}, primero se convirtieron los símbolos génicos a identificadores Entrez mediante la función \texttt{bitr()}.
Posteriormente se aplicó \texttt{enrichKEGG()} con los siguientes parámetros:

\begin{itemize}
	\item \textbf{Organismo:} \texttt{hsa} (\textit{Homo sapiens})
	\item \textbf{Umbral de significancia:} p-valor < 0.05, q-valor < 0.2
\end{itemize}

Los resultados se visualizaron mediante gráficos de puntos (\textit{dotplots}) y gráficos de barras, generados con las funciones \texttt{dotplot()} y \texttt{barplot()} del paquete \texttt{enrichplot}.

Este análisis permitió relacionar los módulos detectados con procesos neuronales y rutas metabólicas asociadas al deterioro neurológico progresivo, en consonancia con los mecanismos descritos en la literatura biomédica \cite{Kelser2024,Gao2008}.

\subsubsection{Exportación y reproducibilidad}

Todos los resultados, incluyendo métricas de nodos, resultados de enriquecimiento y la red en formatos GraphML y GML, se exportaron para su posterior análisis.
El código completo del análisis se documentó en un notebook R Markdown y se encuentra disponible en el repositorio GitHub del proyecto, garantizando la total reproducibilidad del estudio.