\section{Resultados}

\subsection{Red PPI asociada al fenotipo HP:0002344}

La consulta al término HP:0002344 en la Human Phenotype Ontology produjo un conjunto inicial de \textbf{XX genes} asociados al fenotipo. Tras la depuración de duplicados y el mapeo en STRINGdb, \textbf{XX genes} fueron identificados correctamente y empleados para la construcción de la red PPI.

La red final incluyó un total de \textbf{XX nodos} y \textbf{XX interacciones} con un umbral de confianza de 0.7 (combined score $\geq$ 700). La densidad obtenida fue de \textbf{X.XXX}, con un grado medio de \textbf{X.X}. La Tabla~\ref{tab:metricas_globales} resume las métricas topológicas principales de la red.

%\begin{figure}[H]
%	\centering
%	\includegraphics[width=0.85\linewidth]{FIGURA_PPI.png}
%	\caption{Red PPI construida a partir de los genes asociados a HP:0002344 tras el filtrado de interacciones de alta confianza. Los nodos representan genes y las aristas interacciones proteína--proteína.}
%	\label{fig:ppi}
%\end{figure}
%
%\begin{table}[H]
%	\centering
%	\caption{Métricas topológicas globales de la red PPI.}
%	\label{tab:metricas_globales}
%	\begin{tabular}{l c}
%		\hline
%		Métrica & Valor \\
%		\hline
%		Número de nodos & XX \\
%		Número de aristas & XX \\
%		Densidad & X.XXX \\
%		Grado medio & X.X \\
%		Coeficiente de agrupamiento global & X.XXX \\
%		Moduralidad (Louvain) & X.XXX \\
%		\hline
%	\end{tabular}
%\end{table}

\subsection{Distribución de grados y nodos altamente conectados}

La distribución de grados mostró valores comprendidos entre \textbf{X} y \textbf{X}. El nodo con mayor conectividad presentó un grado de \textbf{X}. El histograma de grados obtenido se presenta en la Figura~\ref{fig:histograma_grado}.

%\begin{figure}[H]
%	\centering
%	\includegraphics[width=0.75\linewidth]{FIGURA_HISTOGRAMA_GRADOS.png}
%	\caption{Distribución de grados de la red PPI. El eje X representa el grado y el eje Y la frecuencia.}
%	\label{fig:histograma_grado}
%\end{figure}

\subsection{Detección de comunidades}

La aplicación del algoritmo Louvain detectó un total de \textbf{X comunidades}, con un valor de modularidad de \textbf{X.XXX}. Por su parte, el algoritmo Walktrap identificó \textbf{X comunidades}, alcanzando una modularidad de \textbf{X.XXX}. La Figura~\ref{fig:louvain} muestra la partición obtenida mediante Louvain y la Figura~\ref{fig:walktrap} representa la partición correspondiente a Walktrap.

%\begin{figure}[H]
%	\centering
%	\includegraphics[width=0.8\linewidth]{FIGURA_LOUVAIN.png}
%	\caption{Partición comunitaria generada mediante el algoritmo Louvain. Los colores representan comunidades distintas.}
%	\label{fig:louvain}
%\end{figure}
%
%\begin{figure}[H]
%	\centering
%	\includegraphics[width=0.8\linewidth]{FIGURA_WALKTRAP.png}
%	\caption{Partición comunitaria generada mediante el algoritmo Walktrap. Los colores representan comunidades distintas.}
%	\label{fig:walktrap}
%\end{figure}

\subsection{Métricas individuales y genes de interés}

Las métricas de grado, centralidad de intermediación, cercanía y coeficiente de agrupamiento se calcularon para cada nodo de la red. La Tabla~\ref{tab:metricas_individuales} resume estos valores.

%\begin{table}[H]
%	\centering
%	\caption{Métricas topológicas individuales de los genes analizados.}
%	\label{tab:metricas_individuales}
%	\begin{tabular}{l c c c c}
%		\hline
%		Gen & Grado & Betweenness & Closeness & Clustering \\
%		\hline
%		MECP2 & X & X.XXX & X.XXX & X.XXX \\
%		IARS2 & X & X.XXX & X.XXX & X.XXX \\
%		GRN   & X & X.XXX & X.XXX & X.XXX \\
%		\hline
%	\end{tabular}
%\end{table}

Se generaron subredes de vecindad inmediata (orden 1) para los genes \textit{MECP2}, \textit{IARS2} y \textit{GRN}. La Figura~\ref{fig:ego_mecp2} muestra la subred de MECP2 y las Figuras~\ref{fig:ego_iars2} y~\ref{fig:ego_grn} las correspondientes a los otros genes.
%
%\begin{figure}[H]
%	\centering
%	\includegraphics[width=0.8\linewidth]{FIGURA_EGO_MECP2.png}
%	\caption{Subred de vecindad de orden 1 del gen \textit{MECP2}.}
%	\label{fig:ego_mecp2}
%\end{figure}

\subsection{Enriquecimiento funcional}

El análisis GO identificó \textbf{X términos significativamente enriquecidos} (p ajustado < 0.05). Los términos principales estuvieron relacionados con \textbf{XXX}, \textbf{XXX} y \textbf{XXX}. El análisis KEGG detectó \textbf{X rutas significativas} tras la conversión a identificadores Entrez.

Las Figuras~\ref{fig:go} y~\ref{fig:kegg} muestran los dotplots obtenidos para GO y KEGG, respectivamente.

%\begin{figure}[H]
%	\centering
%	\includegraphics[width=0.75\linewidth]{FIGURA_GO.png}
%	\caption{Dotplot del análisis de enriquecimiento GO (procesos biológicos).}
%	\label{fig:go}
%\end{figure}
%
%\begin{figure}[H]
%	\centering
%	\includegraphics[width=0.75\linewidth]{FIGURA_KEGG.png}
%	\caption{Dotplot del análisis KEGG.}
%	\label{fig:kegg}
%\end{figure}

\subsection{Resumen de los resultados}

En conjunto, el análisis de la red PPI, la detección de comunidades, la caracterización de métricas topológicas y los análisis de enriquecimiento funcional proporcionaron una visión estructural y funcional del conjunto de genes asociados al fenotipo HP:0002344. La integración de estos resultados se desarrolla en la sección de Discusión.

