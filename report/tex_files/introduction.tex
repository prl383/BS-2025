\section{Introducción}

Las enfermedades neurodegenerativas representan uno de los principales retos biomédicos contemporáneos debido a su alta prevalencia y a la ausencia de tratamientos curativos eficaces [1][2]. Aunque sus manifestaciones clínicas son diversas, muchas comparten un rasgo común: la degeneración neurológica progresiva, entendida como un deterioro gradual e irreversible de las funciones neuronales que afecta al movimiento, la cognición y la función sensorial [3].

Este rasgo clínico se encuentra descrito en la Ontología del Fenotipo Humano (HPO) bajo el identificador HP:0002344 – Progressive neurologic deterioration, un término que agrupa un conjunto heterogéneo de aproximadamente 80 enfermedades hereditarias y adquiridas. Entre ellas se incluyen ataxias espinocerebelosas, trastornos mitocondriales, enfermedades lisosomales y síndromes metabólicos del sistema nervioso central [4]. Aunque cada una de estas patologías es ultrarrara (frecuencia <1:1.000.000), su frecuencia agregada se estima en torno a 1 por cada 100.000 personas, lo que convierte a este fenotipo en un problema relevante dentro del espectro de las enfermedades neurológicas raras [5][6].

La HPO constituye una herramienta clave para relacionar fenotipos clínicos con genes causales mediante un vocabulario jerárquico que describe rasgos humanos de forma estandarizada [7]. Esta ontología ha mejorado la interpretación funcional de variantes genéticas y ha facilitado el diagnóstico computacional de enfermedades raras basándose en descripciones fenotípicas precisas [8]. En el caso del deterioro neurológico progresivo, los genes implicados participan en procesos esenciales como el metabolismo energético neuronal, la autofagia, la homeostasis del calcio y la degradación proteica, mecanismos críticos para la supervivencia neuronal [9][6].

Sin embargo, la heterogeneidad genética y fenotípica de las enfermedades asociadas a HP:0002344 dificulta la identificación de mecanismos comunes de patogénesis. En este contexto, la biología de sistemas se ha consolidado como un enfoque prometedor para integrar datos genómicos, transcriptómicos, proteómicos y metabolómicos en modelos de redes moleculares complejas [10][11]. Estas redes, en particular las redes de interacción proteína-proteína (PPI), permiten identificar módulos de genes con funciones compartidas y predecir rutas disfuncionales implicadas en la neurodegeneración [12][13].

El análisis de redes posibilita visualizar las interacciones entre genes implicados en la degeneración neurológica progresiva e identificar nodos clave o hubs cuya alteración podría tener un impacto significativo en la red global [14]. Este enfoque no solo facilita la comprensión de la arquitectura molecular de las enfermedades neurodegenerativas, sino que también contribuye a formular hipótesis sobre procesos celulares alterados y mecanismos de convergencia fenotípica.

A partir de esta base, se plantea la hipótesis de que el fenotipo Progressive neurologic deterioration surge como resultado de alteraciones en redes moleculares interconectadas, compartidas por distintas enfermedades raras con orígenes genéticos diversos. Por tanto, el objetivo de este trabajo es aplicar enfoques de biología de sistemas para modelar y analizar las redes génicas y proteicas asociadas al fenotipo HP:0002344, con el fin de identificar rutas comunes de disfunción, genes candidatos no descritos previamente y posibles nodos terapéuticos susceptibles de intervención.

Dado que este fenotipo es muy amplio y puede manifestarse en un gran número de enfermedades, nos centraremos en un grupo reducido de tres o cuatro patologías representativas, con el objetivo de acotar el análisis y obtener resultados más precisos y manejables.

La integración de información proveniente de bases de datos como HPO, OMIM, STRINGdb y GeneCards permitirá construir una representación global del deterioro neurológico progresivo como un fenotipo sistémico y multifactorial, aportando una base para el desarrollo de estrategias terapéuticas dirigidas que modulen los procesos degenerativos de manera específica [15][16][17].

