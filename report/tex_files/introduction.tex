
\section{Introducción}

Las enfermedades neurodegenerativas constituyen uno de los mayores desafíos biomédicos contemporáneos debido a su alta prevalencia y a la ausencia de tratamientos curativos eficaces \cite{Kelser2024} . Aunque sus manifestaciones clínicas son diversas, comparten un rasgo común: la degeneración neurológica progresiva, entendida como un deterioro gradual e irreversible de las funciones neuronales que afecta al movimiento, la cognición y la función sensorial \cite{Gao2008} .

Este rasgo se encuentra descrito en la Ontología del Fenotipo Humano (HPO) bajo el identificador HP:0002344 – Progressive neurologic deterioration, que agrupa un conjunto heterogéneo de enfermedades hereditarias y adquiridas, entre ellas ataxias, encefalopatías metabólicas y demencias frontotemporales \cite{Wakap2020} . Aunque cada una de estas patologías es ultrarrara (frecuencia <1:1.000.000), su frecuencia combinada alcanza aproximadamente 1 por cada 100.000 personas, lo que convierte a este fenotipo en un problema relevante dentro de las enfermedades neurológicas raras \cite{HPO_term_HP0002344} .

La HPO constituye una herramienta clave para vincular fenotipos clínicos con genes causales mediante un vocabulario jerárquico estandarizado \cite{Wakap2019} . Esta ontología ha mejorado la interpretación de variantes genéticas y ha facilitado el diagnóstico computacional de enfermedades raras basadas en descripciones fenotípicas precisas. 
Sin embargo, la heterogeneidad genética y fenotípica de las enfermedades asociadas a HP:0002344 dificulta la identificación de mecanismos comunes de patogénesis. Para acotar el análisis, nos centraremos en un conjunto representativo de cuatro patologías con este fenotipo: Spinocerebellar ataxia with epilepsy (ORPHA:778), Leigh syndrome (ORPHA:506), Frontotemporal lobar degeneration with TDP-43 inclusions (OMIM:607485) y Rett syndrome (ORPHA:778). Estas patologías ejemplifican distintos mecanismos de neurodegeneración —epigenético, metabólico y proteopático—, permitiendo explorar posibles rutas convergentes en el deterioro neurológico progresivo.

En la ataxia espinocerebelosa con epilepsia y el síndrome de Rett, las mutaciones en MECP2 alteran la regulación epigenética y la expresión génica neuronal. MECP2 codifica una proteína que se une al ADN metilado y modula la transcripción de genes esenciales para la plasticidad sináptica y la maduración neuronal; su disfunción produce desequilibrios en la excitabilidad cortical y pérdida progresiva de funciones cognitivas y motoras.

La enfermedad de Leigh, asociada a mutaciones en IARS2, afecta la función de la isoleucil-ARNt sintetasa mitocondrial, lo que compromete la traducción proteica dentro de la mitocondria y conduce a un déficit energético severo, estrés oxidativo y degeneración neuronal en regiones con alta demanda metabólica.
Por su parte, la degeneración lobar frontotemporal con inclusiones de TDP-43, causada por mutaciones en GRN, se caracteriza por la pérdida de progranulina, una proteína implicada en la homeostasis lisosomal, la inflamación y la supervivencia neuronal. La deficiencia de progranulina conduce a la acumulación de inclusiones citoplasmáticas de TDP-43, disfunción sináptica y neuroinflamación crónica.

A partir de esta base, se plantea la hipótesis de que la degeneración neurológica progresiva surge como resultado de alteraciones en redes moleculares interconectadas, compartidas por distintas enfermedades raras con orígenes genéticos diversos. 

La integración de información procedente de bases de datos como HPO, OMIM, STRINGdb y GeneCards permitirá construir una visión sistémica del fenotipo HP:0002344, identificando redes funcionales compartidas, genes candidatos no descritos y nodos terapéuticos potenciales. Este enfoque de biología de sistemas busca arrojar luz a los mecanismos moleculares de convergencia que subyacen al deterioro neurológico progresivo y aportar una base para el desarrollo de estrategias terapéuticas dirigidas \cite{HPO2025,Ehrhart2016_Rett_MECP2_pathways,Dong2024_IARS2_Leigh} .